%\title{emnlp 2017 instructions}
% File emnlp2017.tex
%

\documentclass[11pt,letterpaper]{article}
\usepackage{emnlp2017}
\usepackage{times}
\usepackage{latexsym}
\usepackage{amsmath}
\usepackage{graphicx}
\usepackage{amssymb}
\usepackage{enumitem}
\usepackage{booktabs}
\usepackage{array}
\newtheorem{theorem}{Theorem}

\usepackage{url}


% Uncomment this line for the final submission:
\emnlpfinalcopy

%  Enter the EMNLP Paper ID here:
\def\emnlppaperid{884}

% To expand the titlebox for more authors, uncomment
% below and set accordingly.
% \addtolength\titlebox{.5in}    

\newcommand\BibTeX{B{\sc ib}\TeX}

\title{Word Embeddings based on Fixed-Size Ordinally Forgetting Encoding}

% Author information can be set in various styles:
% For several authors from the same institution:
% \author{Author 1 \and ... \and Author n \\
%         Address line \\ ... \\ Address line}
% if the names do not fit well on one line use
%         Author 1 \\ {\bf Author 2} \\ ... \\ {\bf Author n} \\
% For authors from different institutions:
% \author{Author 1 \\ Address line \\  ... \\ Address line
%         \And  ... \And
%         Author n \\ Address line \\ ... \\ Address line}
% To start a seperate ``row'' of authors use \AND, as in
% \author{Author 1 \\ Address line \\  ... \\ Address line
%         \AND
%         Author 2 \\ Address line \\ ... \\ Address line \And
%         Author 3 \\ Address line \\ ... \\ Address line}
% If the title and author information does not fit in the area allocated,
% place \setlength\titlebox{<new height>} right after
% at the top, where <new height> can be something larger than 2.25in
\author{Joseph Sanu$^\dagger$, Mingbin Xu$^\dagger$, Hui Jiang$^\dagger$ \and Quan Liu$^\ddagger$\\
$^\dagger$Department of Electrical Engineering and Computer Science, \\
York University, 4700 Keele Street, Toronto, Ontario, M3J 1P3, Canada \\
$^\ddagger$Department of EEIS, University of Science and Technology of China, Hefei, China \\
{\tt \small cse83186@cse.yorku.ca,   xmb@cse.yorku.ca,  hj@cse.yorku.ca, quanliu@mail.ustc.edu.cn}
}

\date{}

\begin{document}

\maketitle

\begin{abstract}
In this paper, we propose to learn word embeddings based on the recent fixed-size ordinally forgetting encoding (FOFE) method, which can almost uniquely encode any variable-length sequence into a fixed-size representation. We use FOFE to fully encode the left and right context of each word in a corpus to construct a novel word-context matrix, which is further weighted and factorized using truncated SVD to generate low-dimension word embedding vectors. We have evaluated this alternative method in encoding word-context statistics and show the new FOFE method has a notable effect on the resulting word embeddings. Experimental results on several popular word similarity tasks have demonstrated that the proposed method outperforms many recently popular neural prediction methods as well as the conventional SVD models that use canonical count based techniques to generate word context matrices.
\end{abstract}


\section{Introduction}

Low dimensional vectors as word representations are very popular in NLP tasks such as inferring semantic similarity and relatedness. Most of these representations are based on either matrix factorization or context sampling described by \cite{baroni2014don} as count or predict models. The basis for both models is the distributional hypothesis \cite{harris1954distributional}, which states that words that appear in similar contexts have similar meaning.
Traditional context representations have been obtained by capturing co-occurrences of words from a fixed-size window relative to the focus word. This representation however does not encompass the entirety of the context surrounding the focus word. Therefore, the distributional hypothesis is not being taken advantage of to the fullest extent.
In this work, we seek to capture these contexts through the fixed-size ordinally forgetting encoding (FOFE) method, recently proposed in \cite{zhang}. In addition to just capturing word co-occurrences, we attempt to use the FOFE to encode the full contexts of each focus word, including the order information of the context sequences. We believe the full encoding of contexts can enhance the resulting word embedding vectors, derived by factoring the corresponding word-context matrix. 
As argued in \cite{zhang}, the FOFE method can almost uniquely encode discrete sequences of varying lengths into a fixed-size code, and this encoding method was used to address the challenges of a limited size window when using deep neural networks for language modeling. The resulting algorithm fulfills the needs of keeping long term dependency while being fast. The word order in a sequence is modeled by FOFE using an ordinally-forgetting mechanism which encodes each position of every word in the sequence.

In this paper, we elaborate how to use the FOFE to fully encode context information of each focus word in text corpora, and present a new method to construct the word-context matrix for word embedding, which may be weighted and factorized as in traditional vector space models \cite{Turney2010}. Next, we report our  experimental results on several popular word similarity tasks, which demonstrate that the proposed FOFE-based approach leads to significantly better performance in these tasks, comparing with the conventional vector space models as well as the popular neural prediction methods, such as {\it word2vec}, {\it GloVe} and more recent {\it Swivel}.  Finally, this paper will conclude 
with the analysis and prospects of combining this approach with other methods.

\section{Related Work}

There has been some debate as to what the optimal length of a text should be for measuring word similarity. Word occurrences from a fixed context window of words can be used to represent a context  \cite{Lund}. The word co-occurrence frequencies are based on fixed windows spanning in both directions from the focus word.  This is then used to create a word-context matrix from which row vectors can be used to measure word similarity. A weighting step is usually applied to highlight words with close association in the co-occurrence matrix, and the truncated SVD is used to factorize the weighted matrix to generate low-dimension word vectors. 
Recently, \cite{mikolov2013efficient} has introduced an alternative way to generate word embeddings using the skipgram model trained with stochastic gradient descent and negative sampling, named as SGNS. SGNS tries to maximize the dot product between $w \cdot c$ where both a word $w$ and a context $c$ are obtained from observed word-context pairs, and meanwhile it also tries to minimize the dot product between $w \cdot c'$ where $c'$ is a negative sample representing some contexts that are not observed in the corpus. More recently, \cite{levy2014} has showed that the objective function of SGNS is essentially seeking to minimize the difference between the model’s estimate and the log of co-occurrence count. Their finding has shown that the optimal solution is a weighted factorization of a pointwise mutual information matrix shifted by the log of the number of negative samples. 

SGNS and GloVe \cite{pennington} select a fixed window of usually 5 words or less around a focus word to encode its context and the word order information within the window is completely ignored. Other attempts to fully capture the contexts have been successful with the use of recurrent neural networks (RNNs) but these methods are much more expensive to run over large corpora when comparing with the proposed FOFE method in this paper. Some previous approaches to encode order information, such as such as BEAGLE \cite{beagle} and Random Permutations \cite{rand}, typically require the use of  expensive operations such as convolution and permutation to process all n-grams within a context window to memorize order information for a given word. On the contrary, the FOFE methods only use a simple recursion to process a sentence once to memorize both context and order information for all words in the sentence. 

\section{FOFE based Embedding}
\label{sec_FOFE}
To capture the full essence of the distributional hypothesis, we need to fully encode the left and right context of each focus word in the text, and further take into accounts that words closer to the focus word should play a bigger role in representing the context relevant to the focus word than other words locating much farther away. 
Traditional co-occurrence word-context matrixes fail to address these concerns of context representation. 

In this work, we propose to make use of the fixed-size ordinally-forgetting encoding (FOFE) method, proposed in  \cite{zhang} as a unique encoding method for any variable-length sequence of discrete words. 
%In \cite{zhang}, this approach has been shown to be quite effective in modeling long term dependency in deep neural networks for several language modeling tasks. 

Given a vocabulary of size $K$, FOFE uses 1-of-K one-hot representation to represent each word. To encode any variable-length sequence of words, FOFE generates the code using a simple recursive formula from the first word ($w_1$) to the last one ($w_T$) of  the sequence: (assume $z_0 = 0$)
\begin{equation}
z_{t}=\alpha \cdot z_{t-1}+e_{t}\ (1 \leq t \leq T)
\end{equation}
where $z_{t}$ denotes the FOFE code for the partial sequence up to word $w_{t}$, $\alpha$ is a constant forgetting factor, and $e_{t}$ denotes the one-hot vector representation of word $w_t$. 
In this case, the code $z_T$ may be viewed as a fixed-size representation of any sequence of $\{w_1, w_2, \cdots, w_T\}$. 
For example, assume we have three symbols in vocabulary, e.g., {\em A}, {\em B}, {\em C}, whose 1-of-K codes are $[1,0,0]$, $[0,1,0]$ and $[0,0,1]$ respectively. When calculating from left to right, the FOFE code for the sequence {\em \{ABC\}} is $[\alpha^2, \alpha ,1]$, and that of {\em \{ABCBC\}} is $[\alpha^4, \alpha+\alpha^3 ,1+\alpha^2]$.

The uniqueness of the FOFE code is made evident if the original sequence can be unequivocally recovered from the given FOFE code. According to \cite{zhang}, FOFE codes have some nice theoretical properties to ensure the uniqueness, as exemplified by the following two theorems \footnote{See \cite{zhang_arxiv} for the proof of these two theorems.}:

\begin{theorem}
\label{theorem-FOFE-alpha-less-half}
If the forgetting factor $\alpha$ satisfies $0<\alpha \leq 0.5$, {\em FOFE} is unique for any $K$ and $T$.	
\end{theorem}

\begin{theorem}
\label{theorem-FOFE-alpha-less-one}
For $0.5 < \alpha <1$, given any finite values of $K$ and $T$,  {\em FOFE} is almost unique everywhere for  $\alpha \in (0.5, 1.0)$, except only a finite set of countable choices of $\alpha$. 	
\end{theorem}

Finally, for alpha values less than or equal to 0.5 and greater than 0, the FOFE is unique for any sequence. For alpha values greater than 0.5, the chance of collision is extremely low and the FOFE is unique in almost all cases. Too find more about the theoretical correctness of FOFE, please refer to \cite{zhang}. In other words, the FOFE codes can almost uniquely encode any sequences, serving as a fixed-size but theoretically lossless representation for any variable-length sequences. 

In this work, we propose to use FOFE to encode the full context where each focus word appears in text. As shown in Figure \ref{fig:FOFE_matrix}, the left context of a focus word, i.e.,  {\em bank}, may be viewed as a sequence and encoded as a  FOFE code $L$ from the left to right while its right context is encoded as another FOFE code $R$ from right to left. When a proper forgetter factor $\alpha$ is chosen, the two FOFE codes can almost fully represent the context of the focus word. If the focus word appears multiple times in text, a pair of FOFE codes $[L, R]$ is generated for each occurrence. Next, a mean vector is calculated for each word from all of its occurrences in text. Finally, as shown in  Figure \ref{fig:FOFE_matrix}, we may line up these mean vectors (one word per row) to form a new word-context matrix, called the FOFE matrix here.

\begin{figure*}[t]
	\centering
	\includegraphics[width=1.0\linewidth]{FOFE-Context-Matrix.pdf} %{FOFE-Context-Matrix2.jpeg}
	\caption{i) encoding left and right contexts of each focus word with FOFE and ii) forming the FOFE word-context matrix.}
	\label{fig:FOFE_matrix}
\end{figure*}

\section{PMI-based Weighting and SVD-based Matrix Factorization}
\label{ssec:SVD}

We further weight the above FOFE matrix using the standard positive pointwise mutual information (PMI)
\cite{Church:90} which has been shown to be of benefit for regular word-context matrices \cite{Pantel:02}.
PMI is used as a measure of association between a word and a context. 
PMI tries to compute the association probabilities based on co-occurrence frequencies. Positive pointwise mutual information is a commonly adopted approach where all negative values in the PMI matrix are replaced with zero. 
The PMI-based weighting function is critical here since it helps to highlight the more surprising events in original word-context matrix.

There are significant benefits in working with low-dimensional dense vectors, as noted by \cite{deerwester1990indexing} with the use of truncated singular value decomposition (SVD). Here, we also use truncated SVD to factorize the above weighted FOFE matrix as the product of three dense matrices $U, \Sigma, V^T$,
where $U$ and $V^T$ have orthonormal columns and $\Sigma$ is a diagonal matrix consisting of singular values. If we select $\Sigma$ to be of rank $d$, its diagonal values represent the top $d$ singular values, and $U_{d}$ can be used to represent all word embeddings with $d$ dimensions where each row represents a word vector.  

\section{Experiments}

\label{ssec:Experiments}

%\subsection{Datasets }
%\label{ssec:Datasets}

We conducted experiments on several popular word similarity data sets and compare our FOFE method with other existing word embedding models in these tasks. In this work, we opt to use five data sets: {\em WordSim353} \cite{finkelstein2001placing}, {\em MEN}  \cite{bruni2012distributional}, {\em Mechanical Turk} \cite{radinsky2011word}, {\em Rare Words} \cite{luong2013} and {\em SimLex-999} \cite{hill2015simlex}. The word similarity performance is  evaluated based on the Spearman rank correlation coefficient obtained by comparing cosine distance between word vectors and human assigned similarity scores.   

For our training data, we use the standard {\em enwiki9} corpus which contains 130 million words. The pre-processing stage includes discarding extremely long sentences, tokenizing, lowercasing and splitting each sentence as a context. Our vocabulary size is chosen to be 80,000 for the most frequent words in the corpus. All words  not in the vocabulary are  replaced with the token {\em \textless         
unk\textgreater}.  In this work, we use a python-based library, called {\em scipy} \footnote{ See \url {http://docs.scipy.org/doc/scipy/reference/}.}, to perform truncated SVD to factorize all word-context matrices. 

\begin{table*}%[h]
%  \small
  \centering
  \caption{\label{results-table} The best achieved performance of various word embedding models on all five examined word similarity tasks.}
  \label{tab:table1} \begin{tabular}{|>{\centering}p{3cm}|>{\centering}p{2cm}>{\centering}p{2cm}>{\centering}p{2cm}>{\centering}p{2cm}>{\centering}p{2cm}|c}
    \hline \bf

%    Method & \bf
%WordSim353 & \bf Bruni et al. MEN & \bf Radinsky et al. Turk & \bf Luong et al. Rare Words & \bf Hill et al. SimLex-999 \tabularnewline
   Method & \bf WordSim353 & \bf MEN & \bf Mech Turk & \bf Rare Words & \bf SimLex-999 \tabularnewline
    \hline \hline
%&&&&&\tabularnewline
     VSM+SVD  & 0.7109 & 0.7130 & 0.6258 & 0.4813 & {\bfseries 0.3866}\tabularnewline 
    \hline
%\hline
% Non-SVD Models &&&&&\tabularnewline &&&&&\tabularnewline

    CBOW & 0.6763 & 0.6768 & 0.6621 & 0.4280 & 0.3549 \tabularnewline 
    GloVe & 0.5873 & 0.6350 & 0.5831 & 0.3934 & 0.2883 \tabularnewline 
    SGNS & 0.7028 & 0.6689 & 0.6187 & 0.4360 & 0.3709 \tabularnewline 
    Swivel & 0.7303 & 0.7246 & {\bfseries 0.7024} & 0.4430 & 0.3323 \tabularnewline \hline \hline
    {\bfseries FOFE+SVD} & {\bfseries 0.7580} & {\bfseries 0.7637} & 0.6525 & {\bfseries 0.5002} &{\bfseries 0.3866} \tabularnewline 
	
\hline
	

  \end{tabular}
\end{table*}

\subsection{Experimental Setup}

\label{ssec:base}

Our first baseline is the conventional vector space model (VSM) \cite{Turney2010}, relying on the PMI-weighted co-occurrence  matrix with dimensionality reduction performed using truncated SVD. The dimension of word vectors is chosen to be 300 and this number is kept the same for all models examined in this paper. Our main goal is to outperform VSM as the model proposed in this paper also uses SVD based matrix factorization. This allows for appropriate comparisons between the different word encoding methods. 

For the purpose of completeness, the other non-SVD based embedding models, mainly the more recent neural prediction methods, are also compared in our experiments. As a result, 
we build the second baseline using the skip-gram model provided by the {\small\verb|word2vec|} software package \cite{mikolov2013efficient}, denoted as SGNS. The word embeddings are generated using the recommended hyper-parameters from \cite{Levy}. Their findings show a larger number of negative samples is preferable and increments on the window size have minimal improvements on word similarity tasks.
In our experiments the number of negative samples is set to 5 and the window size is set to 5. In addition, we set the subsampling rate to $10^{-4}$ and run 3 iterations for training. In adition to SGNS, we also obtained results for CBOW, GloVe \cite{pennington} and Swivel \cite{swivel} models using similar recommended settings. While the window size has a fixed limit in the baseline models, our model does not have a window size parameter as the entire sentence is fully captured as well as distinctions between left and right contexts when generating the FOFE codes. The impact of closer context words is further highlighted by the use of the forgetting factor which is unique to the FOFE based word embedding. 

Finally, we use the FOFE codes to construct the word-context matrix and generate word embedding as described in sections \ref{sec_FOFE} and \ref{ssec:SVD}. Throughout our experiments, we have chosen to use a constant forgetting factor $\alpha=0.7$. There is no significant difference in word similarity scores after experimenting with different $\alpha$ values between $[0.6, 0.9]$ when generating FOFE codes.

We have applied the same hyperparameters to both VSM and FOFE methods and fine-tune them based on the recommended settings provided in \cite{Levy}. Although it has been previously reported that context distribution smoothing \cite{mikolov2013distributed} can provide a net positive effect, it did not yield significant gains in our experiments. On the other hand, the eigenvalue weighting parameter tuning \cite{caron2001experiments} proved to be incredibly effective for some datasets but ineffectual in others. The net benefit however is palpable and we include it for both VSM and FOFE methods. 

\subsection{Results and Discussion}

\label{ssec:Results and Discussion}

The best results of all word embedding models are summarized in Table \ref{results-table} for all five examined data sets, which include the the traditional count based VSM with SVD alongside SGNS using {\small\verb|word2vec|} and our proposed FOFE word embeddings. The most discernible piece of information from the table is that the FOFE method significantly outperforms the traditional count based VSM method on most of these word similarity tasks. The results in Table \ref{results-table} show that substantial gains are obtained by FOFE in {\em WordSim353}, {\em MEN} and {\em Rare Words} data sets. The {\em MEN} dataset shows a 7\% relative improvement over the conventional VSM. 

Among all of these five data sets, the proposed FOFE word embedding significantly outperforms VSM in four tasks while yielding similar performance as VSM in the last data set, i.e. {\em SimLex-999}. FOFE also outperforms all the other models except Swivel in the {\em Mech Turk} dataset. It is important to note that this paper does not state that SVD is obligatory to obtain the best model. 
The FOFE method can be complemented with other models such as Swivel in place of count based encoding methods. It is also theoretically guaranteed that the original sentence is perfectly recoverable from this FOFE code. This theoretical guarantee is clearly missing in previous methods to encode word order information, such as both BEAGLE and Random Permutations. It is evident that overall the FOFE encoding method does achieve significant gains in performance in these word similarity tests over the traditional VSM method that applies the same factorization method. This is substantial as  \cite{Levy} demonstrates that larger window sizes when using SVD does not payoff and the optimal context window is 2. We establish that we can indeed encode more information into our embedding with the FOFE codes. 

In summary, our experimental results show great promise in using the FOFE encoding to represent word contexts for traditional matrix factorization methods. As for future work, the FOFE encoding method may be combined with other popular algorithms, such as Swivel, to  replace the co-occurrence statistics based on a fixed window size.   
\section{Conclusion}

The ability to capture the full context without restriction can play a crucial factor in generating superior word embeddings that excel in NLP tasks. The fixed-size ordinally forgetting encoding (FOFE) has the ability to seize large contexts while discriminating contexts that are farther away as being less significant. 
Conventional embeddings are derived from ambiguous co-occurrence statistics that fail to adequately discriminate contexts words even within the fixed-size window. The FOFE encoding technique trumps other approaches in its ability to procure the state of the art results in several word similarity tasks when combined with prominent factorization practices.

\section*{Acknowledgments}

This work is partially supported by a Discovery Grant from Natural Sciences and Engineering Research Council (NSERC) of Canada, and a research donation from iFLYTEK Co., Hefei, China.

\bibliography{emnlp2017}
\bibliographystyle{emnlp_natbib}

\end{document}
