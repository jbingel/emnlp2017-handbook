%\title{emnlp 2017 instructions}
% File emnlp2017.tex
%

\documentclass[11pt,letterpaper]{article}
\usepackage{emnlp2017}
\usepackage{times}
\usepackage{latexsym}

\usepackage[pdftex]{graphicx}     
\graphicspath{{./figures}}  
\usepackage{caption}  
\usepackage[draft]{fixme}

\usepackage{multirow}

% Uncomment this line for the final submission:
\emnlpfinalcopy

%  Enter the EMNLP Paper ID here:
\def\emnlppaperid{***}

% To expand the titlebox for more authors, uncomment
% below and set accordingly.
% \addtolength\titlebox{.5in}    

\newcommand\BibTeX{B{\sc ib}\TeX}


\title{Word Representations in Factored Neural Machine Translation}

% Author information can be set in various styles:
% For several authors from the same institution:
% \author{Author 1 \and ... \and Author n \\
%         Address line \\ ... \\ Address line}
% if the names do not fit well on one line use
%         Author 1 \\ {\bf Author 2} \\ ... \\ {\bf Author n} \\
% For authors from different institutions:
% \author{Author 1 \\ Address line \\  ... \\ Address line
%         \And  ... \And
%         Author n \\ Address line \\ ... \\ Address line}
% To start a seperate ``row'' of authors use \AND, as in
% \author{Author 1 \\ Address line \\  ... \\ Address line
%         \AND
%         Author 2 \\ Address line \\ ... \\ Address line \And
%         Author 3 \\ Address line \\ ... \\ Address line}
% If the title and author information does not fit in the area allocated,
% place \setlength\titlebox{<new height>} right after
% at the top, where <new height> can be something larger than 2.25in

\author{Franck Burlot\thanks{\hspace{1.5mm}Both authors have contributed equally to this work.} \\ LIMSI, CNRS, Universit\'{e} Paris Saclay
  \And Mercedes Garc\'{i}a-Mart\'{i}nez\footnotemark[1] \\ LIUM, University of Le Mans
  \AND Lo\"{i}c Barrault  \\ LIUM, University of Le Mans
  \And Fethi Bougares \\ LIUM, University of Le Mans
  \AND Fran\c{c}ois Yvon \\ LIMSI, CNRS, Universit\'{e} Paris Saclay}
%% \AND
%%   {\tt firstname.lastname@limsi.fr}
%% \AND
%%    {\tt firstname.lastname@univ-lemans.fr}}\\



\date{}

\begin{document}

\maketitle

\begin{abstract}
  Translation into a morphologically rich language
  requires a large output vocabulary to model various
  morphological phenomena, which is a challenge for
  neural machine translation architectures. To address this issue,
  the present paper investigates the impact of having
  two output factors with a system able to generate
  separately two distinct representations of the target
  words. Within this framework, we investigate
  several word representations that correspond to
  different distributions of morpho-syntactic information
  across both factors. We report experiments for translation
  from English into two morphologically rich languages,
  Czech and Latvian, and show the importance of explicitly
  modeling target morphology.
\end{abstract}


\input{intro}

\input{related_work}

\input{fnmt}

\input{word_representations}

\input{experiments}

\input{aut_eval}

\input{qua_eval}

\input{conclusion}

\section*{Acknowledgments}

This work has been partly funded by the European Union’s
Horizon 2020 research and innovation programme under grant
agreement No.~645452 (QT21) and the French National Research Agency (ANR) through the CHIST-ERA M2CR project, under the contract number ANR-15-CHR2-0006-01. 

\bibliography{biblio}
\bibliographystyle{emnlp_natbib}

\end{document}
