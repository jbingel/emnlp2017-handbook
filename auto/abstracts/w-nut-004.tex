Videogame streaming platforms have become a paramount example of noisy user-generated text. These are websites where gaming is broadcasted, and allows interaction with viewers via integrated chatrooms. Probably the best known platform of this kind is Twitch, which has more than 100 million monthly viewers. Despite these numbers, and unlike other platforms featuring short messages (e.g. Twitter), Twitch has not received much attention from the Natural Language Processing community. In this paper we aim at bridging this gap by proposing two important tasks specific to the Twitch platform, namely (1) Emote prediction; and (2) Trolling detection. In our experiments, we evaluate three models: a BOW baseline, a logistic supervised classifiers based on word embeddings, and a bidirectional long short-term memory recurrent neural network (LSTM). Our results show that the LSTM model outperforms the other two models, where explicit features with proven effectiveness for similar tasks were encoded.
