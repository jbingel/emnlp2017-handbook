In Foreign Language Teaching and Learning (FLTL), questions are systematically used to assess the learner's understanding of a text. Computational linguistic approaches have been developed to generate such questions automatically given a text (e.g., Heilman, 2011). In this paper, we want to broaden the perspective on the different functions questions can play in FLTL and discuss how automatic question generation can support the different uses. Complementing the focus on meaning and comprehension, we want to highlight the fact that questions can also be used to make learners notice form aspects of the linguistic system and their interpretation. Automatically generating questions that target linguistic forms and grammatical categories in a text in essence supports incidental focus-on-form (Loewen, 2005) in a meaning-focused reading task. We discuss two types of questions serving this purpose, how they can be generated automatically; and we report on a crowd-sourcing evaluation comparing automatically generated to manually written questions targeting particle verbs, a challenging linguistic form for learners of English.
