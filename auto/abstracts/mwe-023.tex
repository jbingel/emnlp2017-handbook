Multiword expressions (MWEs) are vexing for linguists, psycholinguists and computational linguists, as they are hard to define, detect and parse. However, previous studies have not taken into account the cognitive constraints under which MWEs are produced or comprehended. We present a new modality for studying MWEs, keystroke dynamics. We ask subjects to respond to a variety of questions, varying in the level of cognitive demand required to generate an answer. In each response, a subject's pause time preceding each word -- within and outside an MWE -- can illuminate distinct differences in required effort across tasks. By taking advantage of high-precision keystroke loggers, we show that MWEs produced under greater cognitive demands are produced more slowly, at a rate more similar to free expressions. We hypothesize that increasingly burdensome cognitive demands diminish the capacity of lexical retrieval, and cause MWE production to slow.
