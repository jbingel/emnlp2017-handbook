In this paper we present the dataset of 200,000+ political arguments produced in the local phase of the 2016 Chilean constitutional process. We describe the human processing of this data by the government officials, and the manual tagging of arguments performed by members of our research group. Afterwards we focus on classification tasks that mimic the human processes, comparing linear methods with neural network architectures. The experiments show that some of the manual tasks are suitable for automatization. In particular, the best methods achieve a 90\% top-5 accuracy in a multi-class classification of arguments, and 65\% macro-averaged F1-score for tagging arguments according to a three-part argumentation model.
