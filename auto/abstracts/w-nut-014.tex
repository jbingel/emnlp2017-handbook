Uyghur is the second largest and most actively used social media language in China. However, a non-negligible part of Uyghur text appearing in social media is unsystematically written with the Latin alphabet, and it continues to increase in size. Uyghur text in this format is incomprehensible and ambiguous even to native Uyghur speakers. In addition, Uyghur texts in this form lack the potential for any kind of advancement for the NLP tasks related to the Uyghur language. Restoring and preventing noisy Uyghur text written with unsystematic Latin alphabets will be essential to the protection of Uyghur language and improving the accuracy of Uyghur NLP tasks. To this purpose, in this work we propose and compare the noisy channel model and the neural encoder-decoder model as normalizing methods.
