Linguistic alignment has emerged as an important property of conversational language and a driver of mutual understanding in dialogue. While various computational measures of linguistic alignment in corpus and experimental data have been devised, a systematic evaluation of them is missing. In this study, we first evaluate the sensitivity and distributional properties of three measures, indiscriminate local linguistic alignment (LLA), Spearman's correlation coefficient (SCC), and repetition decay (RepDecay). Then we apply them in a study of interactive alignment and individual differences to see how well they conform to the Interactive Alignment Model (IAM), and how well they can reveal the individual differences in alignment propensity. Our results suggest that LLA has the overall best performance.
