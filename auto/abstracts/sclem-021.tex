Recently \newcite{wen2015semantically} have proposed a Recurrent Neural Network (RNN) approach to the generation of utterances from dialog acts, and shown that although their model requires less effort to develop than a rule-based system, it is able to improve certain aspects of the utterances, in particular their naturalness. However their system employs generation at the word-level, which requires one to pre-process the data by substituting named entities with placeholders. This pre-processing prevents the model from handling some contextual effects and from managing multiple occurrences of the same attribute. Our approach uses a character-level model, which unlike the word-level model makes it possible to learn to ``copy'' information from the dialog act to the target without having to pre-process the input. In order to avoid generating non-words and inventing information not present in the input, we propose a method for incorporating prior knowledge into the RNN in the form of a weighted finite-state automaton over character sequences. Automatic and human evaluations show improved performance over baselines on several evaluation criteria. \emph{To SCLeM reviewers: this ``Extended Abstract'' submission was previously published in Coling-2016.}
