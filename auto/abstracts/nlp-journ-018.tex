This paper discusses the problem of incongruent headlines: those which do not accurately represent the information contained in the article with which they occur. We emphasise that this phenomenon should be considered separately from recognised problematic headline types such as clickbait and sensationalism, arguing that existing natural language processing (NLP) methods applied to these related concepts are not appropriate for the automatic detection of headline incongruence, as an analysis beyond stylistic traits is necessary. We therefore suggest a number of alternative methodologies that may be appropriate to the task at hand as a foundation for future work in this area. In addition, we provide an analysis of existing data sets which are related to this work, and motivate the need for a novel data set in this domain.
