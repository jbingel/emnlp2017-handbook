This paper addresses the task of identifying the bias in news articles published during a political or social conflict. We create a silver-standard corpus based on the actions of users in social media. Specifically, we reconceptualize bias in terms of how likely a given article is to be shared or liked by each of the opposing sides. We apply our methodology to a dataset of links collected in relation to the Russia-Ukraine Maidan crisis from 2013-2014. We show that on the task of predicting which side is likely to prefer a given article, a Naive Bayes classifier can record 90.3\% accuracy looking only at domain names of the news sources. The best accuracy of 93.5\% is achieved by a feed forward neural network. We also apply our methodology to gold-labeled set of articles annotated for bias, where the aforementioned Naive Bayes classifier records 82.6\% accuracy and a feed-forward neural networks records 85.6\% accuracy.
