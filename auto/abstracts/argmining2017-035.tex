Argumentative text has been analyzed both theoretically and computationally in terms of argumentative structure that consists of argument components (e.g., claims, premises) and their argumentative relations (e.g., support, attack). Less emphasis has been placed on analyzing the semantic types of argument components. We propose a two-tiered annotation scheme to label claims and premises and their semantic types in an online persuasive forum, Change My View, with the long-term goal of understanding what makes a message persuasive. Premises are annotated with the three types of persuasive modes: ethos, logos, pathos, while claims are labeled as interpretation, evaluation, agreement, or disagreement, the latter two designed to account for the dialogical nature of our corpus. We aim to answer three questions: 1) can humans reliably annotate the semantic types of argument components? 2) are types of premises/claims positioned in recurrent orders? and 3) are certain types of claims and/or premises more likely to appear in persuasive messages than in non-persuasive messages?
