This paper describes TwitterHawk, a system for sentiment analysis of tweets which participated in the SemEval-2015 Task 10, Subtasks A through D. The system performed competitively, most notably placing 1st in topic-based sentiment classification (Subtask C) and ranking 4th out of 40 in identifying the sentiment of sarcastic tweets. Our submissions in all four subtasks used a supervised learning approach to perform three-way classification to assign positive, negative, or neutral labels. Our system development efforts focused on text pre-processing and feature engineering, with a particular focus on handling negation, integrating sentiment lexicons, parsing hashtags, and handling expressive word modifications and emoticons. Two separate classifiers were developed for phrase-level and tweet-level sentiment classification. Our success in aforementioned tasks came in part from leveraging the Subtask B data and building a single tweet-level classifier for Subtasks B, C and D.
