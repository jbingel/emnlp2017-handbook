We propose a method for the annotation of Japanese civil judgment documents, with the purpose of creating flexible summaries of these. The first step, described in the current paper, concerns content selection, i.e., the question of which material should be extracted initially for the summary. In particular, we utilize the hierarchical argument structure of the judgment documents. Our main contributions are a) the design of an annotation scheme that stresses the connection between legal points (called issue topics) and argument structure, b) an adaptation of rhetorical status to suit the Japanese legal system and c) the definition of a linked argument structure based on legal sub-arguments. In this paper, we report agreement between two annotators on several aspects of the overall task.
