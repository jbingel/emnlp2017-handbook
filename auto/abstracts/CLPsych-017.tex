Mental illnesses, such as depression and post traumatic stress disorder (PTSD), are highly underdiagnosed globally. Populations sharing similar demographics and personality traits are known to be more at risk than others. In this study, we characterise the language use of users disclosing their mental illness on Twitter. Language-derived personality and demographic estimates show surprisingly strong performance in distinguishing  users that tweet a diagnosis of depression or PTSD from random controls, reaching an area under the receiver-operating characteristic curve -- AUC -- of around .8 in all our binary classification tasks. In fact, when distinguishing users disclosing depression from those disclosing PTSD, the single feature of estimated age shows nearly as strong performance (AUC = .806) as using thousands of topics (AUC =.819) or tens of thousands of n-grams (AUC =.812). We also find that differential language analyses, controlled for demographics, recover many symptoms associated with the mental illnesses in the clinical literature.
