The paper presents first results of an ongoing project on text simplification focusing on linguistic metaphors. Based on an analysis of a parallel corpus of news text professionally simplified for different grade levels, we identify six types of simplification choices falling into two broad categories: preserving metaphors or dropping them. An annotation study on almost 300 source sentences with metaphors (grade level 12) and their simplified counterparts (grade{\textasciitilde}4) is conducted. The results show that most metaphors are preserved and when they are dropped, the semantic content tends to be preserved rather than dropped, however, it is reworded without metaphorical language. In general, some of the expected tendencies in complexity reduction, measured with psycholinguistic variables linked to metaphor comprehension, are observed, suggesting good prospect for machine learning-based metaphor simplification.
