For brands, gaining new customer is more expensive than keeping an existing one. Therefore, the ability to keep customers in a brand is becoming more challenging these days.  Churn happens when a customer leaves a brand to another competitor. Most of the previous work considers the problem of churn prediction using the Call Detail Records (CDRs). In this paper, we use micro-posts to classify customers into churny or non-churny. We explore the power of convolutional neural networks (CNNs) since they achieved state-of-the-art in various computer vision and NLP applications. However, the robustness of end-to-end models has some limitations such as the availability of a large amount of labeled data and uninterpretability of these models. We investigate the use of CNNs augmented with structured logic rules to overcome or reduce this issue. We developed our system called Churn\_teacher by using an iterative distillation method that transfers the knowledge, extracted using just the combination of three logic rules, directly into the weight of the DNNs. Furthermore, we used weight normalization to speed up training our convolutional neural networks. Experimental results showed that with just these three rules, we were able to get state-of-the-art on publicly available Twitter dataset about three Telecom brands.
