The use of language to convey emotional experience is of significant importance to the process of psychotherapy, the diagnosis of problems with emotion and memory, and more generally to any communication that aims to evoke a feeling in the recipient.              Bucci's theory of the referential process (1997) concerns three phases whereby a person activates emotional, or bodily experience (Arousal Phase), conveys the images and events associated with it (Symbolizing Phase), and reconsiders them (Reorganizing Phase). The Symbolizing Phase is the major focus of this study and is operationalized by a measure called referential activity (RA) based on judges' ratings of the concreteness, specifcity, clarity and imagery of language style.              Computational models of RA have previously been created in several languages, however, due to the complexity of modeling RA, different modeling strategies have been employed for each language and common features that predict RA across languages are not well understood.  Working from previous computational models developed in English and Italian, this study specifies a new model of predictors common to both languages that correlates between r = .36 and.45 with RA.  The components identified support the construct validity of the referential process and may facilitate the development of measures in other languages.
