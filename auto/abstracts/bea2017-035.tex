This paper is concerned with the task of automatically assessing the written proficiency level of non-native (L2) learners of English. Drawing on previous research on automated L2 writing assessment following the Common European Framework of Reference for Languages (CEFR), we investigate the possibilities and difficulties of deriving the CEFR level from short answers to open-ended questions, which has not yet been subjected to numerous studies up to date. The object of our study is twofold: to examine the intricacy involved with both human and automated CEFR-based grading of short answers. On the one hand, we describe the compilation of a learner corpus of short answers graded with CEFR levels by three certified Cambridge examiners. We mainly observe that, although the shortness of the answers is reported as undermining a clear-cut evaluation, the length of the answer does not necessarily correlate with inter-examiner disagreement. On the other hand, we explore the development of a soft-voting system for the automated CEFR-based grading of short answers and draw tentative conclusions about its use in a computer-assisted testing (CAT) setting.
