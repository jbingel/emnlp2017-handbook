Accessing historical texts is often a challenge because readers either do not know the historical language, or they are challenged by the technological hurdle when such texts are available digitally.Merging corpus linguistic methods and digital technology can provide novel ways of representing historical texts digitally and providing a simpler access. In this paper, we describe a multi-dimensional parallel Old Occitan-English corpus, in which word alignment serves as the basis for search capabilities as well as for the transfer of annotations.  We show how parallel alignment can help overcome some challenges of historical manuscripts. Furthermore, we apply a resource-light method of building an emotion annotation via parallel alignment, thus showing that such annotations are possible without speaking the historical language. Finally, using visualization tools, such as ANNIS and GoogleViz, we demonstrate how the emotion analysis can be queried and visualized dynamically in our parallel corpus, thus showing that such information can be made accessible with low technological barriers.
