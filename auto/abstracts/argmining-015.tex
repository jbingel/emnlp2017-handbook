Argument extraction is the task of identifying arguments, along with their components in text. Arguments can be usually decomposed into a claim and one or more premises justifying it. The proposed approach tries to identify segments that represent argument elements (claims and premises) on social Web texts (mainly news and blogs) in the Greek language, for a small set of thematic domains, including articles on politics, economics, culture, various social issues, and sports. The proposed approach exploits distributed representations of words, extracted from a large non-annotated corpus. Among the novel aspects of this work is the thematic domain itself which relates to social Web, in contrast to traditional research in the area, which concentrates mainly on law documents and scientific publications. The huge increase of social web communities, along with their user tendency to debate, makes the identification of arguments in these texts a necessity. In addition, a new manually annotated corpus has been constructed that can be used freely for research purposes. Evaluation results are quite promising, suggesting that distributed representations can contribute positively to the task of argument extraction.
