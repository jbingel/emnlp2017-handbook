We propose a novel method to learn negation expressions in a specialized (medical) domain. In our corpus, negations are annotated as ‘flat' text spans. This allows for some infelicities in the mark-up of the ground truth, making it less than perfectly aligned with the underlying syntactic structure. Nonetheless, the negations thus captured are correct in intent, and thus potentially valuable. We succeed in training a model for detecting the negated predicates corresponding to the annotated negations, by re-mapping the corpus to anchor its ‘flat' annotation spans into the predicate argument structure. Our key idea—re-mapping the negation instance spans to more uniform syntactic nodes—makes it possible to re-frame the learning task as a simpler one, and to leverage an imperfect resource in a way which enables us to learn a high performance model. We achieve high accuracy for negation detection overall, 87\%. Our re-mapping scheme can be constructively applied to existing flatly annotated resources for other tasks where syntactic context is vital.
