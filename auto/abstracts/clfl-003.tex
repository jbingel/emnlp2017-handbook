This paper investigates how literature could be used as a means to expand our understanding of history. By applying macroanalytic techniques we are aiming to investigate how women enter literature and particularly which functions do they assume, their working patterns and if we can spot differences in how often male and female characters are mentioned with various types of occupational titles (vocation) in Swedish literary texts. Modern historiography, and especially feminist and women's history has emphasized a relative invisibility of women's work and women workers. The reasons to this are manifold, and the extent, the margin of error in terms of women's work activities is of course hard to assess. Therefore, vocation identification can be used as an indicator for such exploration and we present a hybrid system for automatic annotation of vocational signals in 19th century Swedish prose fiction. Beside vocations, the system also assigns gender (male, female or unknown) to the vocation words, a prerequisite for the goals of the study and future in-depth explorations of the corpora.
