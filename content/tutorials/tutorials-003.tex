\begin{bio}
\textbf{Fragkiskos D. Malliaros} is currently a data science postdoctoral scholar in the Department of Computer Science and Engineering at UC San Diego and member of the Artificial Intelligence group. Right before that, he was a postdoctoral researcher in Ecole Polytechnique, France from where he also received his Ph.D. degree in 2015. He obtained his Diploma and his M.Sc. degree from the University of Patras, Greece in 2009 and 2011 respectively. He is the recipient of the 2012 Google European Doctoral Fellowship in Graph Mining and the 2015 Thesis Prize by Ecole Polytechnique. During the summer of 2014, he was a research intern at the Palo Alto Research Center (PARC), working on anomaly detection in social networks. His research interests span the broad area of data science, with focus on graph mining, social network analysis, applied machine learning and natural language processing.

\textbf{Michalis Vazirgiannis} is a Professor in Ecole Polytechnique, France and the leader of the Data Science and Mining (DaSciM) team. He holds a degree in Physics, a M.Sc. in Robotics, both from University of Athens, Greece, and a M.Sc. in Knowledge Based Systems from Heriot-Watt University (Edinburgh, UK). He acquired his Ph.D. degree from the Dept. of Informatics, University of Athens. He has worked as a researcher in different places: NTUA, GMD-IPSI (currently Frauhofer-IPSI), Germany Fern-Universitaet Hagen, in project VERSO (later GEMO) in INRIA/Paris, in IBM India Research Laboratory and in MPI fur Informatik (Saarbruecken, Germany). He held a Marie Curie Intra-European fellow in area of P2P Web Search, hosted by INRIA FUTURS, Paris. His current research interests are on graph mining, text mining and recommendation algorithms. He is chairing the ‘‘AXA Data Science’’ chair in Ecole Polytechnique and has collaborations with the industry including Google and Airbus.
\end{bio}

\begin{tutorial}
  {Graph-based Text Representations: Boosting Text Mining, NLP and Information Retrieval with Graphs}
  {tutorial-final-003}
  {\daydateyear, \tutorialmorningtime}
  {\TutLocC}

Graphs or networks have been widely used as modeling tools in Natural Language Processing (NLP), Text Mining (TM) and Information Retrieval (IR). Traditionally, the unigram bag-of-words representation is applied; that way, a document is represented as a multiset of its terms, disregarding dependencies between the terms. Although several variants and extensions of this modeling approach have been proposed (e.g., the n-gram model), the main weakness comes from the underlying term independence assumption. The order of the terms within a document is completely disregarded and any relationship between terms is not taken into account in the final task (e.g., text categorization). Nevertheless, as the heterogeneity of text collections is increasing (especially with respect to document length and vocabulary), the research community has started exploring different document representations aiming to capture more fine-grained contexts of co-occurrence between different terms, challenging the well-established unigram bag-of-words model. To this direction, graphs constitute a well-developed model that has been adopted for text representation. The goal of this tutorial is to offer a comprehensive presentation of recent methods that rely on graph-based text representations to deal with various tasks in NLP and IR. We will describe basic as well as novel graph theoretic concepts and we will examine how they can be applied in a wide range of text-related application domains.
All the material associated to the tutorial will be available at: \url{http://fragkiskosm.github.io/projects/graph_text_tutorial}

\end{tutorial}
