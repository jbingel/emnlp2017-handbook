%\thispagestyle{myheadings}
\section{Keynote Address: Lillian Lee}
\index{Lee, Lillian}

\begin{center}
\begin{Large}
{\bfseries\Large ``Big data pragmatics!'', or, ``Putting the ACL in computational social science'', or, if you think these title alternatives could turn people on, turn people off, or otherwise have an effect, this talk might be for you}\vspace{1em}\par
\end{Large}

%% \begin{center}
%%   \begin{tabular}{m{1in}b{1in}}
%%     \includegraphics[width=1in]{content/monday/cortes-headshot.png}
%%     & {\bfseries Corinna Cortes} \newline Google Research, NY
%%   \end{tabular}
%% \end{center}

\daydateyear, 9:00--10:10am \vspace{1em}\\
\PlenaryLoc \\
\vspace{1em}\par
%\includegraphics[height=100px]{content/monday/cortes-headshot.png}
\end{center}

\noindent
{\bfseries Abstract:} What effect does language have on people?

You might say in response, "Who are you to discuss this problem?" and
you would be right to do so; this is a Major Question that science has
been tackling for many years. But as a field, I think natural language
processing and computational linguistics have much to contribute to
the conversation, and I hope to encourage the community to further
address these issues.

This talk will focus on the effect of phrasing, emphasizing aspects
that go beyond just the selection of one particular word over
another. The issues we'll consider include: Does the way in which
something is worded in and of itself have an effect on whether it is
remembered or attracts attention, beyond its content or context? Can
we characterize how different sides in a debate frame their arguments,
in a way that goes beyond specific lexical choice (e.g., "pro-choice"
vs. "pro-life")? The settings we'll explore range from movie quotes
that achieve cultural prominence; to posts on Facebook, Wikipedia,
Twitter, and the arXiv; to framing in public discourse on the
inclusion of genetically-modified organisms in food.

Joint work with Lars Backstrom, Justin Cheng, Eunsol Choi, Cristian
Danescu-Niculescu-Mizil, Jon Kleinberg, Bo Pang, Jennifer Spindel, and
Chenhao Tan.

\vspace{3em}\par 

\vfill
\noindent

{\bfseries Biography:} Lillian Lee is a professor of computer science
and of information science at Cornell University, and the
co-Editor-in-Chief, together with Michael Collins, of Transactions of
the ACL. Her research interests include natural language processing
and computational social science. She is the recipient of the
inaugural Best Paper Award at HLT-NAACL 2004 (joint with Regina
Barzilay), a citation in “Top Picks: Technology Research Advances of
2004” by Technology Research News (also joint with Regina Barzilay),
and an Alfred P. Sloan Research Fellowship; and in 2013, she was named
a Fellow of the Association for the Advancement of Artificial
Intelligence (AAAI). Her group’s work has received several mentions in
the popular press, including The New York Times, NPR’s All Things
Considered, and NBC’s The Today Show, and one of her co-authored
papers was publicly called “boring” by Youtubers Rhett and Link, in a
video viewed over 1.8 million times.

\newpage
