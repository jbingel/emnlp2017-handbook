%\thispagestyle{myheadings}
\section{Keynote Address: Fei-Fei Li}
\index{Li, Fei-Fei}

\begin{center}
\begin{Large}
{\bfseries\Large ``A Quest for Visual Intelligence in Computers''}
\vspace{1em}\par
\end{Large}

\daydateyear, 9:00--10:10am \vspace{1em}\\
\PlenaryLoc \\
\vspace{1em}\par
%\includegraphics[height=100px]{content/tuesday/popovic-headshot.jpg}
\end{center}

\noindent
{\bfseries Abstract:} More than half of the human brain is involved in
visual processing. While it took mother nature billions of years to
evolve and deliver us a remarkable human visual system, computer
vision is one of the youngest disciplines of AI, born with the goal of
achieving one of the loftiest dreams of AI. The central problem of
computer vision is to turn millions of pixels of a single image into
interpretable and actionable concepts so that computers can understand
pictures just as well as humans do, from objects, to scenes,
activities, events and beyond. Such technology will have a fundamental
impact in almost every aspect of our daily life and the society as a
whole, ranging from e-commerce, image search and indexing, assistive
technology, autonomous driving, digital health and medicine,
surveillance, national security, robotics and beyond. In this talk, I
will give an overview of what computer vision technology is about and
its brief history. I will then discuss some of the recent work from my
lab towards large scale object recognition and visual scene story
telling. I will particularly emphasize on what we call the "three
pillars" of AI in our quest for visual intelligence: data, learning
and knowledge. Each of them is critical towards the final solution,
yet dependent on the other. This talk draws upon a number of projects
ongoing at the Stanford Vision Lab. 

\vspace{3em}\par 

\vfill
\noindent

{\bfseries Biography:} 
Dr.\ Fei-Fei Li is an Associate Professor in the Computer Science
Department at Stanford, and the Director of the Stanford Artificial
Intelligence Lab and the Stanford Vision Lab. Her research areas are
in machine learning, computer vision and cognitive and computational
neuroscience, with an emphasis on Big Data analysis. Dr. Fei-Fei Li
has published more than 100 scientific articles in top-tier journals
and conferences, including Nature, PNAS, Journal of Neuroscience,
CVPR, ICCV, NIPS, ECCV, IJCV, IEEE-PAMI, etc. Dr. Fei-Fei Li obtained
her B.A. degree in physics from Princeton in 1999 with High Honors,
and her PhD degree in electrical engineering from California Institute
of Technology (Caltech) in 2005. She joined Stanford in 2009 as an
assistant professor, and was promoted to associate professor with
tenure in 2012. Prior to that, she was on faculty at Princeton
University (2007-2009) and University of Illinois Urbana-Champaign
(2005-2006). Dr. Fei-Fei Li is a speaker at TED2015 main conference, a
recipient of the 2014 IBM Faculty Fellow Award, 2011 Alfred Sloan
Faculty Award, 2012 Yahoo Labs FREP award, 2009 NSF CAREER award, the
2006 Microsoft Research New Faculty Fellowship and a number of Google
Research awards. Work from Fei-Fei’s lab have been featured in a
number of popular press magazines and newspapers including New York
Times, Wired Magazine, and New Scientists.

\newpage
